
Conducting a study, without commenting and discussing on the results produced, does not make much os a sense. On Chapter \ref{sec:Experiment} (section \ref{sec:hypotheses}) we stated the hypotheses we made regarding this study. Then, on Chapter \ref{sec:Results} we presented, analysed and commented on all collected results, after separating them in six groups (Task Completion Time, Offsets, etc.). Now, we need to provide the necessary infrastructure to glue them together. That can be done by commenting and discussing if the hypotheses hold, taking into account the results we calculated. This is what this chapter is about; discuss, comment and reason whether the hypotheses stated, hold or not. Below, we are restating each of the hypotheses, along with discussing on their validity. 
 
\begin{itemize}

    %%%%%%%%
    %%%%%%%%
    %%%%%%%%
    \item \textbf{\textit{(H1). Feedback supplied trials (Discrete or Continuous) outperform No Feedback ones in terms of offset. }}

    In section \ref{sec:resultsOffset}, we mentioned that offset represents the error for each trial. The offset represents the distance (in percentage) between the point we selected and the center of the target. While No Feedback trials have no limitations on the value of the error, in Feedback \& Continuous trials the maximum allowed offset is \emph{+-2\%}.

    In chapter 6 we noted that, in No Feedback trials error rates were huge compared to the FeedBack Trials (only Feedback \& Continuous being considered). This was expected as in No Feedback trials, users have no indication on where is the feedback line; being also free, due to the QuickRelease technique, to lift their finger at whichever point (even if they are outside the target). Their ability to develop memory and predict correctly, was not enough to outperform Feedback trials. We can then conclude that, H1 holds.

    %%%%%%%%
    %%%%%%%%
    %%%%%%%%
    \item \textbf{\textit{(H2). No Feedback \& Discrete outperforms No Feedback \& Continuous in terms of offset.}}
    
    In section \ref{sec:resultsOffset} we mention that:
    \emph{Offset values for Discrete targeting are not always considered as errors, as we measure offset from the center of the targets range. The real error can be calculated by the type:}   $RealErrorDisrete = |OffsetValue| - (TargetRange*0.5)$

    On the results section, we noticed, much of our interest, that both NFD and NFC share the same error rates. Taking into account the aforementioned statement, that the Real Error in NFD trials is always smaller than offset we are provided with, we indeed realize that the real error in NFD trials is smaller than in NFC ones. As a result, H2 holds.

    %%%%%%%%
    %%%%%%%%
    %%%%%%%%
    \item \textbf{\textit{(H3). Task completion time will be gradually decreased over time.}}

    In other words, our hypothesis states that: 

    $AverageTimeRep1 > AverageTimeRep2 > AverageTimeRep3$ 

    In section \ref{sec:LCtotalTime} we presented and analysed the Task Completion Time Learning Curve. Learning curve is directly connected and related with the task completion time improvements over time. Taking into account the graphs presented in figure \ref{fig:learningCurveTotalTime}, we realize that there exist obvious learning improvements from repetition 1 to repetition 3, thus task completion time gradually decreases over time.  The only type of trials that we do not observe continuous improvement over time, is Feedback \& Continuous, for which in the 3rd repetition we have a slight increase of the average task completion time. As we also mentioned in section \ref{sec:LCtotalTime}, mental and physical fatigue might have a share in this behaviour, concluding that if it was not for fatigue (mental and physical), task completion time might have been further decreased on repetition 3. We can then conclude that hypothesis 3 for most type of trials.

    %%%%%%%%
    %%%%%%%%
    %%%%%%%%
    \item \textbf{\textit{(H4). Error rates will be gradually decreased over time.}}
    
    In section \ref{sec:LCOffset}, and more specific in Figure \ref{fig:learningCurveOffset} we observed that for all FC, NFD, NFC trials there exists a minor but almost unobservable error improvement over time, except probably for NFD. We observed that error rates slightly dropped on the second repetition, and remained stable or decreased in some cases on the third one. We can then conclude that, H4 is supported from the results, and thus it holds in general. 


    %%%%%%%%
    %%%%%%%%
    %%%%%%%%
    \item \textbf{\textit{(H5). Task competition time, when feedback is provided, is dependent on the number of elements.}}

    In section \ref{sec:resultsTaskCompletionTime} we made the following observations. When both Feedback and No Feedback trials are concerned, task completion time does not statistically differ for \emph{N = 2, 3, 4, 6, 8}. However we do observe a difference when $N=12$ or $N=16$. Thus H5 does not hold for all four different kinds of trials.
    However, addressing Feedback trials only, the scenery is alternated. For instance, for Feedback \& Discrete, completion time increases as the number of Elements increases. we observe $1.460 s$ for $N=2$, ending up with \emph{3.858 s} for $N=16$. As we have already mentioned, in Discrete trials, the number of elements (N) is highly related with the size of the target. However this does not hold for Continuous trials since there, N specifies the range of the position of the target. The size of a continuous target is fixed, and smaller than $N=16$ targets for Discrete trials. We indeed observed FC the completion time is stable and independent on the number of elements (\emph{N}), which makes sense exactly because in FC target size does not even change. Thus for FC trials, time is independent on the number of the elements, but is dependent of the size of the elements. Thus H5 holds only for FD trials, while if we modify it a bit it can also hold for FC trials. A modified version of H5 would be:

    \emph{Task competition time, when feedback is provided, is dependent on the \textbf{size of the target}.}




    %%%%%%%%
    %%%%%%%%
    %%%%%%%%
    \item \textbf{\textit{(H6). Average contact areas will be subconsciously preferred by users, as this is the natural position of the finger.}}
    
    If we observe Figure \ref{fig:meanOffset} in section \ref{sec:resultsOffset} we notice that in No Feedback trials, offset rates follow a common pattern. We experienced high positive offsets for targets located close to minimum contact area, almost zero offset when the target was located closed to the average contact area, and high negative offsets when target was located close to the maximum contact area. That, in other words means that, participants when not limited by the feedback line indicator, tent to subconsciously prefer and select contact areas that are closer to the average - median contact area. However when feedback is provided, they are forced by the application to precisely select the predefined target. That makes it impossible to study what H6 hypothesizes when we deal with Feedback trials. We then conclude, that when Feedback is not provided H6 holds.


    %%%%%%%%
    %%%%%%%%
    %%%%%%%%
    \item \textbf{\textit{(H7). Feedback \& Discrete is the most preferred type of trial, as it best combines speed and accuracy.}}

    In the additional assessment users provided us with (Appendix \ref{assessmentFatFinger}), we asked them to put the 4 different techniques in ascending order depending on which was the easiest to use. Users definitely preferred Feedback\&Discrete trials (as the easiest to use), with Feedback\&Continuous, No Feedback\&Discrete and Feedback\&Continuous following on the line. This definitely depicts user preferences, and if we also observe the results we got from the offset and completion time parameters, we realize that Feedback\&Discrete outperforms all other techniques regarding all measured parameters. Even for Task Completion time, in which due to the Dwell technique it requires an 1s delay to confirm target selection, the results are comparable to No Feedback trials, which in turn do not require any extra delays. As a result we can positively support that H7 holds.

\end{itemize}




Taking into consideration the above discussed hypotheses we perceive that all hypotheses hold, with minor modifications at certain cases. We note that \emph{Feedback \& Discrete} seems to be the most effective out of all types of trials, having comparable task completion time with No Feedback trials, almost overpassing the $1s$ obligatory delay obstacle. As discussed in section \ref{sec:resultsTaskCompletionTime}, task completion time does not statistically differ for N = 2, 3, 4, 6, 8. Thus we can conclude that users are able to distinguish up to 8 different discrete pressure levels when feedback is supplied, without any statistical significant difference on the task completion time. H7 further supports our claim, stating that FD is the most preferred selection technique, which also combines the best values for accuracy with a correspondent speed. 

We then observed that completion time for No Feedback trials is independent on the number of elements, $N$. Thus, performance can be mainly measured through the error rates. In section \ref{sec:LCOffset} we mentioned and commented on the error rates and the corresponding learning factor. 
We concluded that a level of 3, possibly 4, discrete levels is in turn identifiable and distinguishable by users, when feedback is not provided. 
H6 reveals that average contact areas are subconsciously preferred by users. This mainly happens because the natural position of the finger on the screen, requires an average contact size. This tendency is a significant parameter which contributes to the ability of No feedback trials to distinguish less discrete pressure levels; excluding the non-presence of the feedback region, which is the main and most important one.
We finally saw that the ability of a person to develop haptic memory to precisely select a target when N is very high, is limited. This further explains why the performance on No Feedback trials is decreased compared to the Feedback ones. 



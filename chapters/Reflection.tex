\section{Reflection}

\subsection{Perspective and Assumptions on Software Development}

The four paradigms identified in \cite{Refl} establish assumptions which information systems development approaches, as SRUM is, follow. Actually in \cite{Refl} is said that Development approaches, as SCRUM is, are influenced from assumptions that belong to more than one paradigms. As a result, our project, as I will explain below, mainly fits under Functionalism realm, but shares aspects from Social Relativism too.

Our system development proceeded from "without", by applying formal concepts through planned intervention. Daily Scrum meeting and Backlog head towards that. Furthermore, our role as developers were that of an expert - platonic philosopher, assisting the product owner fulfill his goal, scope and vision. We guided him to the final solution, after him giving us the requirements. Moreover, we gave him the ability to view things from a perspective, in which he did not need to know many details to understand the whole concept.
On the other hand we should take into account that product owner was not acting as a "dictator". Thus his reality wasn’t the only true one. He used to communicate with us as if we were partners. There were many times we all together tried to comprehend the different realities, seeking the one that made the most sense. That behavior leads us into Social Relativism realm. Also throughout the project  it seemed that we all had quite the same rights. We all had speech, in what we were producing. Moreover, in Social Relativism scrum demo enables evolutionary learning, since it allows for the project to be accepted and reduces change resistance by being embedded into the social perception and sense-making process \cite{Refl}. Finally, Social Relativism, is accordant with SCRUM's self-organizing teams as was done during the Review-Planning Meetings.

To sum up our process has mostly embraced Functionalism. But since we worked together with the product owner to reach a shared understanding of what we were doing, or in other words, our team was able-allowed to provide input on the actual features of the application; we also touched Social Relativism.

\subsection{Perspective and assumptions inherent in SCRUM}
Scrum  defines "a flexible, holistic product development strategy where a development team works as a unit to reach a common goal" as opposed to a "traditional, sequential approach". It enables teams to self organize and adopts an empirical approach. It accepts that the problem cannot be fully understood or defined and instead focuses on maximizing the team's ability to deliver quickly and respond to emerging requirements.
There are three core roles in SCRUM: Product Owner, Development Team and the Scrum Master. And three types of meetings: Sprint planning meeting, Daily Scrum meeting and the End meetings (Sprint Review and Sprint Retrospective).
 
But what makes Scrum different from all other approaches?

 Usually before we start developing any actual software, we’re making a lot of assumptions about what functionality we’re going to need. We're assuming we understand the problem structure and also the solution. We’re assuming what end users are looking for, what they really need - want and what makes them use our product. We’re assuming the time it will take to deliver functionality, what is technically possible or impossible and so on. Every single one of these assumption affects the success of our project. If our assumptions are wrong we will fail, and indeed the reality is that most of our assumptions will turn out to be wrong.
 
This is the point where Scrums joins the game. It allows us to quickly and frequently (Scrum meetings) check our assumptions in order to prevent expensive failures and deliver working software every sprint, because \emph{The sooner you know that your assumptions are wrong, the more options you have}. And when that happens, you can either drop the functionality, ignore it, change it or extend it. So Scrum provides us with a number of means to identify incorrect assumptions, with the sprint review being the most obvious one. So, The goal of the "game" is to optimize and control the delivery of Valuable Software. Control is achieved via frequent Inspect \& Adapt cycles and the ability of the members to learn and improve, to become better players. In general, SCRUM requires great discipline, but leaves much room for personal creativity. It is based upon respect for the people through a neat and balanced separation of responsibilities.
 



In this study we proposed Fat Finger, an alternative interaction technique that uses the finger's contact size as the main source of input on a mobile or tablet device. The main objective was to investigate how many distinguishable contact size levels can be actually achieved and perceived by users. 
Current environment and developments on the market were discussed in Chapter \ref{sec:Introduction}, along with explanation of the limitations of the 2D interaction techniques, used to interact with current smartphones and tablet devices. 
Then we broached the 3D interaction techniques, whose principle is to use the contact size of the finger touching the screen, in addition to it's position on the screen. 
Relevant work and researches that have been already conducted on the same or highly related fields, were in turn presented in chapter \ref{sec:RelatedWork}. We separated it in pressure-based and contact-shape-based approaches, thoroughly explaining similarities and differences with Fat Finger. 

In Chapter \ref{sec:FatFingerConcept} we presented the original idea behind Fat Finger interaction technique, along with the way we conceptualized its usage. Finally, we raised the basic research question that this study will respond to.
 \emph{"To which extend are we able to distinguish the different simulated pressure levels produced by our fingers using a tablet device?"}
The overall concept is that we want to investigate the capabilities and precision of using the contact area of the index finger to interact with a tablet device. 

The interface, the structure and the design of the application we developed to test and study Fat Finger interaction technique, was reported in chapter \ref{sec:Implementation}. It is in the form of discrete target selection trials, runs on and Apple iPad device, and consists of many consequent trials, each of them requiring the user to perform a target selection task.
Firstly, we explained the basic work-flow we meet once we launch the application, followed by the basic design and interface of a simple trial. Then we analyse the 4 categories of trials: Feedback \& Discrete, Feedback \& Continuous, No Feedback \& Discrete and  No Feedback \& Continuous. Finally, we explain the final sequence of the trials in the user study and the procedure and methods we used to monitor and measure user performance.
In Chapter \ref{sec:Experiment} we continue by analysing the procedure and the context of the user study. We had 26 participants taking part in this study.  Each user had to go through the verbal instructions, fill in the Demographic information form, calibrate his finger, accomplish the six hundred and twelve trials of the experiment, and finally fill in an assessment for each of the 4 different type of trials. We then gave statistical information regarding the demographic information of the participants that took place in our study. 

Finally, in Chapter \ref{sec:Results} we presented the computed results for all the basic parameters: Task completion time, Offset, Task Completion Time Learning Curve, Offset Learning Curve, Re-Entries and Re-Touches. Discussion on the results and arguing on the validity of the hypotheses stated in Chapter \ref{sec:Experiment}, are all performed in Chapter \ref{sec:Discussion}. We conclude that for Feedback supplied trials a level of 8 distinguishable contact size levels is easily perceived by users, while for No-Feedback ones we meet tolerated error rates only until 3 or 4 contact size levels. Also, we gave evidence and supported that all of the hypotheses we have set, hold.

While this study focuses on understanding touch and contact size on mobile devices, much work has to be done in order to produce real life products that will make use of this interaction technique. It will require the redesign of current interfaces, to make the cooperation with this technique feasible. But prior to that, we need to find the most appropriate way(s) to interact and exploit these multiple level contact size capabilities.

Thus, next steps and future work will include the seek of a way to integrate this technique into on-market products. Fat Finger is capable to become both a sufficient alternative to the current gestured based approaches and also a much intuitive way to perform operations that are infeasible at the moment. 
An immediate way to integrate interaction through contact size with mobile interfaces is probably feasible by allowing the control of continuous variables through contact size variation. For instance, volume, brightness, exposure etc., will now be controlled without the usage of a bar, which will also not occupy any on-screen space. As a result, interfaces will become simpler and more intuitive.
Another approach, would be to use our finger for performing precise operations, for which currently, only pens are used. For the shake of illustration, S Pen is used by Samsung to perform operations that need augmented precision and detail. Is it possible using Fat Finger, to achieve the same kind of detail through an interface which will provide us the necessary infrastructure?
Finally, another suggestion is to study how Fat Finger can be applied and integrated with multi-finger interactions. What is the impact on the ability to perceive the contact size levels, when multiple finger are used? 
Which are the possible ways to combine current gestures with Fat Finger interaction technique?

The aforementioned proposed techniques suggest feasible ways to take the Fat Finger concept to the next level; they might be embedded in future mobile applications, or other techniques might be unveiled through extensive research. 
However, the assimilation degree Fat Finger will encounter depends highly on the level of satisfaction, user experience, pleasure and throughput it will incorporate. In the end, we must accept the reality that the future of Fat Finger will depend largely on market trends, which are mainly determined by human beings, their needs and desires.
 


% Lets hope , contact size based input will become a standard on every day mobile interaction. 


